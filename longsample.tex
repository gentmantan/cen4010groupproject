%%
%% This is file `./samples/longsample.tex',
%% generated with the docstrip utility.
%%
%% The original source files were:
%%
%% apa7.dtx  (with options: `longsample')
%% ----------------------------------------------------------------------
%% 
%% apa7 - A LaTeX class for formatting documents in compliance with the
%% American Psychological Association's Publication Manual, 7th edition
%% 
%% Copyright (C) 2021 by Daniel A. Weiss <daniel.weiss.led at gmail.com>
%% 
%% This work may be distributed and/or modified under the
%% conditions of the LaTeX Project Public License (LPPL), either
%% version 1.3c of this license or (at your option) any later
%% version.  The latest version of this license is in the file:
%% 
%% http://www.latex-project.org/lppl.txt
%% 
%% Users may freely modify these files without permission, as long as the
%% copyright line and this statement are maintained intact.
%% 
%% This work is not endorsed by, affiliated with, or probably even known
%% by, the American Psychological Association.
%% 
%% ----------------------------------------------------------------------
%% 
%% Notes on the paper
%% My specific issue: ethical issues regarding the exploitation of mineral mining for the use of electronics production
\documentclass[stu]{apa7}

\usepackage{lipsum}

\usepackage[american]{babel}

\usepackage{csquotes}
\usepackage[style=apa,backend=biber]{biblatex}

%\usepackage{multirow}

\usepackage{tabularx}
\usepackage{xltabular}

\usepackage{lipsum}

\newenvironment{enumnumbers}
  {\begin{enumerate}[label=\# \arabic{*} :]}
  {\end{enumerate}}

\newcommand{\nextitem}{\par\hspace*{\labelsep}\textbullet\hspace*{\labelsep}}
\newcommand{\nextitemblank}{\par\hspace*{\labelsep}\hspace*{\labelsep}}


\addbibresource{bibliography.bib}

\title{PassUSB}

\authorsnames{Team Number 10: Sonali Benni, Rey Jairus Marasigan, Chidiebere Otuonye, Kaiya Roberts, Gentman Tan}
\authorsaffiliations{Florida International University}
\course{CEN4010 U02: Software Engineering I}
\professor{Kianoosh G. Boroojeni}
\duedate{March 11, 2022}

\abstract{}

\begin{document}
\maketitle

\section{Introduction}

With the prevalence of information technologies, there exists an ever-increasing need for individuals to secure one's own access to online accounts. The typical method of doing so requires the user to create a secret passphrase that they would then be responsible for memorizing in order to access a given system. However, several factors make such a task difficult and unsafe; first, the exponential rise in computational power has led to the feasibility of "brute force attacks", in turn forcing IT administrators to enforce increased password length and complexity. Another effect that the increase in password length has is making the memorization of multiple different passwords difficult, thereby incentivizing individuals to unsafely reuse their own passwords. With these shortcomings in mind, we propose a system that would solve all of these issues in a single package.

\subsection{Purpose of system}
Solve the security issues facing PC end users in the realm of password authentication

\subsection{Scope of the system}
In scope: Multi-platform mobile application, password management, multi-plaform USB keyboard emulation, mitigations against man-in-the-middle, replay and spoofing attacks
Out of scope: Application data security, side-channel attacks

\subsection{Objectives and success criteria of the project}

\begin{APAitemize}
  \item Provide a mobile app for users to create and store passwords
  \item Provide a USB hardware dongle that can be paired with the mobile app which can type in passwords in lieu of keyboard input
\end{APAitemize}

\subsection{Definitions, acronyms, and abbreviations}

\begin{APAitemize}
  \item PassUSB: the project's USB dongle solution that emulates a USB keyboard
  \item Password manager: a computer program that generates, stores and retrieves passswords for its users
  \item HID (Human Interface Device): a computer device that facilitates communications between a computer user and a computer 
  \item App: a computer application
  \item Pairing: the process of recognition and acknowledgement between the mobile device and USB dongle
\end{APAitemize}

\subsection{Overview of document}
The project will consist of two types of coding assignments, one for frontend development i.e. mobile app development, and the other for backend development i.e. microcontroller programming. The mobile app will prompt the user to create a new password database, in which he/she will then enter a master password that is to be used to secure the database. The user will be given an option to pair the PassUSB with the app. Should the user choose to or not choose to pair the PassUSB, the user is then able to utilize the app's password generation, management and storage features.

%%Chapter 3 here 

\subsection{Requirements Elicitation}

\begin{xltabular}{\textwidth}{|l|X|}

  \hline Use Case ID: & UC-232 \\ \hline
  Created By: Gentman Tan & Last Updated By: Gentman Tan \\ \hline
  Actors: & \nextitem User \nextitem MobileDevice \nextitem PassUSB \\ \hline
  Description: & Initialize a secure connection between the Owner's MobileDevice to the PassUSB for future device communication \\ \hline
  Preconditions: & \nextitem The User has created a database \nextitem The User has opened the app and unlocked their database \nextitem The User has selected the menu option to pair a new device\\ \hline
  Normal Flow: &
  \nextitem User is prompted to enable Bluetooth discovery mode 
  \nextitem Once found, the Bluetooth dongle's name and serial number is displayed
  \nextitem User is prompted to either allow or decline the pairing process with the displayed bluetooth dongle
  \nextitem When the "Allow" option is chosen, the MobileDevice and PassUSB is subsequently paired \\ \hline

\end{xltabular}

\scriptsize{\begin{xltabular}{\textwidth}{|l|X|}

  \hline Use Case ID: & UC-1-AA \\ \hline
  Use Case Name: & Add account \\ \hline
  Created By: Rey Jairus Marasigan & Last Updated By: Rey Jairus Marasigan \\ \hline
  Date Created: February 21, 2022 & Last Revision Date: February 21, 2022 \\ \hline
  Actors: & \nextitem User \\ \hline
  Description: &  This use case is used for specifying the different elements of our system that the DeviceUser goes through to add an entry to a list of account entries within our app. The ideal outcome of this use case is that the account is added to the database without any adverse effects. \\ \hline
  Trigger: & The user presses the plus icon on the top right of the UI. \\ \hline
  Preconditions: & \nextitem The User has created a database \nextitem The User has opened the app and unlocked their database \nextitem The User is in the correct directory to be able to see and touch the plus icon. \\ \hline
  Postconditions: & \nextitem At minimum: the system retains consistency and integrity in case of an error or deviation from the flow of events i.e. the user can “add account” again without repercussions from previous try.
    \nextitem Everything in harmony and works without error: Account is added to the database.
    \nextitem The account added can be accessed after being saved to the database. \\ \hline
  Normal Flow: & 
  (assumes the user wants to generate password when adding the account)
  \begin{enumerate}
    \item The DeviceUser presses the plus icon on the top right
    \item The view switches to a display that shows 3 text fields: a website field, a username and a password field. The first two will be required to be filled using the pop-up keyboard. The password field can either be manually filled or automatically generated using the associated button located to its left. <generate password use case>
    \item The user presses a button that opens to another view that shows multiple fields, boxes, and sliders to be filled, checked, and adjusted, respectively, to meet a certain criteria associated with the password.
    \item The user enters the field with a pop-up keyboard on their phone.
    \item Below the fields, the users checks various boxes or switches that configures the password generated by the apps such as:
      \nextitem contains special characters- an option that includes special characters in the password
      \nextitem contains numbers- an option that includes number
      \nextitem etc.
  \item Below the switches, there exists a slider that controls how long the password will be. When it is adjusted towards the right, it increases the length. When it is adjusted towards the left, it decreases the length. The maximum length of the slider occupies the entire width of the screen with some deadzone. The minimum length is 8 characters and the maximum is 32 characters. Initially the slider is set to 12.
  \item As the user changes the length of the desired password, a text box that shows the password itself changes. The password is shown in its pure form i.e. the actual password to be used. Here, the password is changeable by pressing on the box and using the pop-up keyboard. Initially, the password generated here is from the combination of the slider and checkboxes.
  \item The user then presses the "save password" button below to generate the password. <generate password use case>
  \item The user finally then presses the “add account” button below to add the account to the list.
  \item A notification pops up that lets the user know that the account has been added.
  \item The state of the app then returns to its default view.
  \end{enumerate} \\ \hline
 Alternative Flows: & \nextitemblank 2b. In step 2 of the normal flow, if the user chooses to manually enter the password rather than generate a new password,
   \begin{enumerate}
     \item The entire normal flow can skip to step 9
   \end{enumerate}
   \nextitemblank 7b. In step 7 of the normal flow, if the user changes the generated password and then slides the password slider,
   \begin{enumerate}
     \item An entirely different password is generated.
     \item The new password is instantly displayed within the text box.
     \item The normal use case continues thereafter on step 7 where the user can edit the newly generated password.
   \end{enumerate}
   \nextitemblank 9b. In step 9 of the normal flow, if the website field and username field is not filled
     \begin{enumerate}
     \item A notification pops up to let the user know about the required fields
     \item The app is scrolled the field needed to be filled
     \item The fields are highlighted red to indicate requirement. 
     \item The keyboard pops up automatically
     \item Once the fields are filled, the use case enters step 8 of the normal flow.
     \end{enumerate} \\ \hline
   Exceptions: & Exceptions to adding account to list
     \nextitemblank 9b. In step 9 of the normal flow, if the user does not enter a website or username. \\ \hline
   Includes: & editAccount- steps 3 to 7 would be required to edit an entry within the list and can be separated in its own use case \\ \hline
   Frequency of Use: & On-demand\\ \hline
   Special Requirements: & \nextitem Users should not wait >20 seconds for an action to be performed
     \nextitem Animations should last <250ms
     \nextitem Colors should be made to be accessible to colorblind
     \nextitem Application should be made available on all major mobile platforms \\ \hline
   Assumptions: & \nextitem The Customer has installed and is using the application
     \nextitem The application is readable to the Customer
     \nextitem The client has the ability to use the keyboard and interact with the screen. \\ \hline
   Notes and Issues: & \nextitem Is the minimum and maximum possible length of the password generator feasible for the users?
     \nextitem Should there be a limit to the amount of accounts added to the list?
     \nextitem What are the different switches that we can include in the forum?
     \nextitem should the app request for the master password (the password required to enter the app)or other biometrics for the ability to add an entry to the list?
     \nextitem Should we use other characters outside of the printable ascii characters? \\ \hline
\end{xltabular}



\scriptsize{\begin{xltabular}{\textwidth}{|l|X|}
  \hline Use Case ID: & UC-2-AB \\ \hline
  Use Case Name: & Login to Database \\ \hline
  Created By: Sonali Benni & Last Updated By: Sonali Benni \\ \hline
  Date Created: February 22, 2022 & Last Revision Date: February 22, 2022 \\ \hline
  Actors: & \nextitem DeviceUser \\ \hline
  Description: & This use case is for logging into the app to access passwords. Ideally, the actor would have the app on their phone. They would log into the app and have access to their home screen/ main menu. \\ \hline
  Trigger: & The DeviceUser clicks on the app on their personal device. \\ \hline
  Preconditions: & \nextitem The DeviceUser is logged in on their personal device such as a cellphone where the app is located. \nextitem The DeviceUser opens the app \\ \hline
  Postconditions: & \nextitem The DeviceUser is able to get into the app and is brought to a screen that allows them to enter login information.
    \nextitem The DeviceUser has successfully entered their information and is brought to a screen that shows their home screen. 
  Normal Flow: & 
   Alternative Flows: & \nextitemblank 2b. In step 2 of the normal flow, if the user chooses to manually enter the password rather than generate a new password,
   \begin{enumerate}
     \item The entire normal flow can skip to step 9
   \end{enumerate}
   \nextitemblank 7b. In step 7 of the normal flow, if the user changes the generated password and then slides the password slider,
   \begin{enumerate}
     \item An entirely different password is generated.
     \item The new password is instantly displayed within the text box.
     \item The normal use case continues thereafter on step 7 where the user can edit the newly generated password.
   \end{enumerate}
   \nextitemblank 9b. In step 9 of the normal flow, if the website field and username field is not filled
     \begin{enumerate}
     \item A notification pops up to let the user know about the required fields
     \item The app is scrolled the field needed to be filled
     \item The fields are highlighted red to indicate requirement. 
     \item The keyboard pops up automatically
     \item Once the fields are filled, the use case enters step 8 of the normal flow.
     \end{enumerate} \\ \hline
   Exceptions: & Exceptions to adding account to list
     \nextitemblank 9b. In step 9 of the normal flow, if the user does not enter a website or username. \\ \hline
   Includes: & editAccount- steps 3 to 7 would be required to edit an entry within the list and can be separated in its own use case \\ \hline
   Frequency of Use: & On-demand\\ \hline
   Special Requirements: & \nextitem Users should not wait >20 seconds for an action to be performed
     \nextitem Animations should last <250ms
     \nextitem Colors should be made to be accessible to colorblind
     \nextitem Application should be made available on all major mobile platforms \\ \hline
   Assumptions: & \nextitem The Customer has installed and is using the application
     \nextitem The application is readable to the Customer
     \nextitem The client has the ability to use the keyboard and interact with the screen. \\ \hline
   Notes and Issues: & \nextitem Is the minimum and maximum possible length of the password generator feasible for the users?
     \nextitem Should there be a limit to the amount of accounts added to the list?
     \nextitem What are the different switches that we can include in the forum?
     \nextitem should the app request for the master password (the password required to enter the app)or other biometrics for the ability to add an entry to the list?
     \nextitem Should we use other characters outside of the printable ascii characters? \\ \hline
\end{xltabular}



\end{document}}

%% 
%% Copyright (C) 2021 by Daniel A. Weiss <daniel.weiss.led at gmail.com>
%% 
%% This work may be distributed and/or modified under the
%% conditions of the LaTeX Project Public License (LPPL), either
%% version 1.3c of this license or (at your option) any later
%% version.  The latest version of this license is in the file:
%% 
%% http://www.latex-project.org/lppl.txt
%% 
%% Users may freely modify these files without permission, as long as the
%% copyright line and this statement are maintained intact.
%% 
%% This work is not endorsed by, affiliated with, or probably even known
%% by, the American Psychological Association.
%% 
%% 
%% This work is "maintained" (as per LPPL maintenance status) by
%% Daniel A. Weiss.
%% 
%% This work consists of the file  apa7.dtx
%% and the derived files           apa7.ins,
%%                                 apa7.cls,
%%                                 apa7.pdf,
%%                                 README,
%%                                 APA7american.txt,
%%                                 APA7british.txt,
%%                                 APA7dutch.txt,
%%                                 APA7english.txt,
%%                                 APA7french.txt,
%%                                 APA7german.txt,
%%                                 APA7ngerman.txt,
%%                                 APA7greek.txt,
%%                                 APA7czech.txt,
%%                                 APA7turkish.txt,
%%                                 APA7endfloat.cfg,
%%                                 Figure1.pdf,
%%                                 shortsample.tex,
%%                                 longsample.tex, and
%%                                 bibliography.bib.
%% 
%%
%% End of file `./samples/longsample.tex'.
