%%
%% This is file `./samples/longsample.tex',
%% generated with the docstrip utility.
%%
%% The original source files were:
%%
%% apa7.dtx  (with options: `longsample')
%% ----------------------------------------------------------------------
%% 
%% apa7 - A LaTeX class for formatting documents in compliance with the
%% American Psychological Association's Publication Manual, 7th edition
%% 
%% Copyright (C) 2021 by Daniel A. Weiss <daniel.weiss.led at gmail.com>
%% 
%% This work may be distributed and/or modified under the
%% conditions of the LaTeX Project Public License (LPPL), either
%% version 1.3c of this license or (at your option) any later
%% version.  The latest version of this license is in the file:
%% 
%% http://www.latex-project.org/lppl.txt
%% 
%% Users may freely modify these files without permission, as long as the
%% copyright line and this statement are maintained intact.
%% 
%% This work is not endorsed by, affiliated with, or probably even known
%% by, the American Psychological Association.
%% 
%% ----------------------------------------------------------------------
%% 
%% Notes on the paper
%% My specific issue: ethical issues regarding the exploitation of mineral mining for the use of electronics production
\documentclass[stu]{apa7}

\usepackage{lipsum}

\usepackage[american]{babel}

\usepackage{csquotes}
\usepackage[style=apa,backend=biber]{biblatex}

\newenvironment{enumnumbers}
  {\begin{enumerate}[label=\# \arabic{*} :]}
  {\end{enumerate}}

\addbibresource{bibliography.bib}

\title{PassUSB}

\authorsnames{Team Number 10: Sonali Benni, Rey Jairus Marasigan, Chidiebere Otuonye, Kaiya Roberts, Gentman Tan}
\authorsaffiliations{Florida International University}
\course{CEN4010 U02: Software Engineering I}
\professor{Kianoosh G. Boroojeni}
\duedate{March 11, 2022}

\abstract{}

\begin{document}
\maketitle

With the prevalence of information technologies, there exists an ever-increasing need for individuals to secure one's own access to online accounts. The typical method of doing so requires the user to create a secret passphrase that they would then be responsible for memorizing in order to access a given system. However, several factors make such a task difficult and unsafe; first, the exponential rise in computational power has led to the feasibility of "brute force attacks", in turn forcing IT administrators to enforce increased password length and complexity. Another effect that the increase in password length has is making the memorization of multiple different passwords difficult, thereby incentivizing individuals to unsafely reuse their own passwords. With these shortcomings in mind, we propose a system that would solve all of these issues in a single package.

\subsection{Purpose of system}
Solve the security issues facing PC end users in the realm of password authentication

\subsection{Scope of the system}
In scope: Multi-platform mobile application, password management, multi-plaform USB keyboard emulation, mitigations against man-in-the-middle, replay and spoofing attacks
Out of scope: Application data security, side-channel attacks

\subsection{Objectives and success criteria of the project}

\begin{APAitemize}
  \item Provide a mobile app for users to create and store passwords
  \item Provide a USB hardware dongle that can be paired with the mobile app which can type in passwords in lieu of keyboard input
\end{APAitemize}

\subsection{Definitions, acronyms, and abbreviations}

\begin{APAitemize}
  \item PassUSB: the project's USB dongle solution that emulates a USB keyboard
  \item Password manager: a computer program that generates, stores and retrieves passswords for its users
  \item HID (Human Interface Device): a computer device that facilitates communications between a computer user and a computer 
  \item App: a computer application
  \item Pairing: the process of recognition and acknowledgement between the mobile device and USB dongle
\end{APAitemize}

\subsection{Overview of document}
The project will consist of two types of coding assignments, one for frontend development i.e. mobile app development, and the other for backend development i.e. microcontroller programming. The mobile app will prompt the user to create a new password database, in which he/she will then enter a master password that is to be used to secure the database. The user will be given an option to pair the PassUSB with the app. Should the user choose to or not choose to pair the PassUSB, the user is then able to utilize the app's password generation, management and storage features.

\printbibliography

\end{document}

%% 
%% Copyright (C) 2021 by Daniel A. Weiss <daniel.weiss.led at gmail.com>
%% 
%% This work may be distributed and/or modified under the
%% conditions of the LaTeX Project Public License (LPPL), either
%% version 1.3c of this license or (at your option) any later
%% version.  The latest version of this license is in the file:
%% 
%% http://www.latex-project.org/lppl.txt
%% 
%% Users may freely modify these files without permission, as long as the
%% copyright line and this statement are maintained intact.
%% 
%% This work is not endorsed by, affiliated with, or probably even known
%% by, the American Psychological Association.
%% 
%% 
%% This work is "maintained" (as per LPPL maintenance status) by
%% Daniel A. Weiss.
%% 
%% This work consists of the file  apa7.dtx
%% and the derived files           apa7.ins,
%%                                 apa7.cls,
%%                                 apa7.pdf,
%%                                 README,
%%                                 APA7american.txt,
%%                                 APA7british.txt,
%%                                 APA7dutch.txt,
%%                                 APA7english.txt,
%%                                 APA7french.txt,
%%                                 APA7german.txt,
%%                                 APA7ngerman.txt,
%%                                 APA7greek.txt,
%%                                 APA7czech.txt,
%%                                 APA7turkish.txt,
%%                                 APA7endfloat.cfg,
%%                                 Figure1.pdf,
%%                                 shortsample.tex,
%%                                 longsample.tex, and
%%                                 bibliography.bib.
%% 
%%
%% End of file `./samples/longsample.tex'.
